\documentclass{beamer}

\usepackage[czech,english]{babel}
\usepackage[utf8]{inputenc}
\usepackage[T1]{fontenc}
\usepackage{pgfplots}
\pgfplotsset{compat=1.13}
\usepackage{lmodern}
\usepackage{tikz}
\usetikzlibrary{shapes,arrows,positioning,matrix,backgrounds}
\usepackage{tikz-uml}
\usepackage[orientation=portrait,size=a0,scale=1.5]{beamerposter}
  
\title[]{Paralelní trénování hlubokých neuronových sítí}
\author[]{Bc. Ondřej Šlampa}
\institute[]{Fakulta informačních technologií}
\date{\today}
\usetheme{Pittsburgh}

\begin{document}
\begin{frame}{}

\tikzstyle{node_input}=[draw, fill=white, circle, minimum height=1em, minimum width=1em]
\tikzstyle{node_dot}=[draw, fill=black, circle, inner sep=0em, text width=0.1em]
\tikzstyle{node_neuron}=[draw, fill=white, circle, minimum height=3em, minimum width=3em]
\tikzstyle{node_point}=[coordinate]

\tikzstyle{block_conv}=[draw, fill=white, rectangle, minimum height=2em, minimum width=5em]
\tikzstyle{block_fc}=[draw, fill=white, rectangle, minimum height=2em, minimum width=5em]
\tikzstyle{block_norm}=[draw, fill=lightgray, rectangle, minimum height=2em, minimum width=5em]
\tikzstyle{block_relu}=[draw, fill=lightgray, rectangle, minimum height=2em, minimum width=5em]
\tikzstyle{block_softmax}=[draw, fill=white, rectangle, minimum height=2em, minimum width=5em]
\tikzstyle{block_concat}=[draw, fill=lightgray, rectangle, minimum height=2em, minimum width=5em]
\tikzstyle{block_maxpool}=[draw, fill=lightgray, rectangle, minimum height=2em, minimum width=5em]
\tikzstyle{block_avgpool}=[draw, fill=lightgray, rectangle, minimum height=2em, minimum width=5em]
\tikzstyle{block_sum}=[draw, fill=white, circle, minimum height=2em, minimum width=2em]
\tikzstyle{block_sc}=[coordinate]
\tikzstyle{block_sc2}=[draw, fill=lightgray, rectangle, minimum height=2em, minimum width=5em, align=left]
\tikzstyle{block_dropout}=[draw, fill=lightgray, rectangle, minimum height=2em, minimum width=5em]
\tikzstyle{block_point}=[coordinate]
\tikzstyle{block_softmax}=[draw, fill=white, ellipse, minimum height=2em, minimum width=5em]

\tikzstyle{block_server}=[draw, fill=white, rectangle, minimum height=2em, minimum width=5em]
\tikzstyle{block_worker}=[draw, fill=white, rectangle, minimum height=2em, minimum width=5em]

\tikzstyle{node_op}=[draw, fill=white, rounded rectangle, minimum height=3em, minimum width=5em]


\begin{beamercolorbox}{}
\maketitle
\end{beamercolorbox}

\begin{block}{Synchronní trénování}

Hlavní částí mé práce je vytvoření způsobu jak odhadovat výhodnost požití distribuovaného trénování pro danou síť.
To je založeno na tom, že trénování je možné rozdělit na výpočet gradientů a komunikaci.
Délku výpočtu gradientů je možné naměřit na jedné jednotce.
Délka komunikace je možné vypočítat podle počtu vah.
Z těchto dvou hodnot je možné vypočítat délku celého trénování.

\begin{columns}[t]

\begin{column}{0.48\linewidth}
\begin{block}{Nejhorší rozložení komunikace}

Obě jednotky komunikují se serverem současně, to způsobí, že rychlost přenosu dat bude poloviční a délka přenosu dvojnásobná.

\begin{figure}
\centering
\scalebox{0.8}{
\begin{tikzpicture}[align=left]
\begin{umlseqdiag}
    \umlobject[class=Zařízení,fill=white,x=0]{Pracovní jednotka 1}
    \umlobject[class=Zařízení,fill=white,x=12]{Server}
    \umlobject[class=Zařízení,fill=white,x=24]{Pracovní jednotka 2}
    \begin{umlfragment}[type=iterace, name=iter, inner xsep=2]
    \begin{umlcall}[op={přijmi váhy()},with return,dt=10,padding=13]{Pracovní jednotka 1}{Server}
    \end{umlcall}
    \begin{umlcall}[op={přijmi váhy()},with return,dt=10,padding=13]{Pracovní jednotka 2}{Server}
    \end{umlcall}
    \begin{umlcallself}[op={trénuj()},with return,dt=3,padding=3,fill=gray]{Pracovní jednotka 1}
    \end{umlcallself}
    \begin{umlcallself}[op={trénuj()},with return,dt=3,padding=3,fill=gray]{Pracovní jednotka 2}
    \end{umlcallself}
    \begin{umlcall}[op={odešli gradienty()},with return,dt=3,padding=13,fill=lightgray]{Pracovní jednotka 1}{Server}
    \end{umlcall}
    \begin{umlcall}[op={odešli gradienty()},with return,dt=3,padding=13,fill=lightgray]{Pracovní jednotka 2}{Server}
    \end{umlcall}
    \end{umlfragment}
    \end{umlseqdiag}
\end{tikzpicture}
}
\end{figure}

\end{block}
\end{column}

\begin{column}{0.48\linewidth}
\begin{block}{Nejlepší rozložení komunikace}

Při tomto rozložení dochází k maximalizaci prokládání výpočtů a komunikace.
Pokud jedna pracovní jednotka komunikuje se serverem, druhá počítá nebo čeká.

\begin{figure}
\centering
\scalebox{0.8}{
\begin{tikzpicture}[align=left]
\begin{umlseqdiag}
    \umlobject[class=Zařízení,fill=white,x=0]{Pracovní jednotka 1}
    \umlobject[class=Zařízení,fill=white,x=12]{Server}
    \umlobject[class=Zařízení,fill=white,x=24]{Pracovní jednotka 2}
    \begin{umlfragment}[type=iterace, name=iter, inner xsep=2]
    \begin{umlcall}[op={přijmi váhy()},with return,dt=10,padding=5]{Pracovní jednotka 1}{Server}
    \end{umlcall}
    \begin{umlcall}[op={přijmi váhy()},with return,dt=19,padding=5]{Pracovní jednotka 2}{Server}
    \end{umlcall}
    \begin{umlcallself}[op={trénuj()},with return,dt=3,padding=3,fill=gray]{Pracovní jednotka 1}
    \end{umlcallself}
    \begin{umlcallself}[op={trénuj()},with return,dt=3,padding=3,fill=gray]{Pracovní jednotka 2}
    \end{umlcallself}
    \begin{umlcall}[op={odešli gradienty()},with return,dt=3,padding=5,fill=lightgray]{Pracovní jednotka 1}{Server}
    \end{umlcall}
    \begin{umlcall}[op={odešli gradienty()},with return,dt=3,padding=5,fill=lightgray]{Pracovní jednotka 2}{Server}
    \end{umlcall}
    \end{umlfragment}
    \end{umlseqdiag}
\end{tikzpicture}
}
\end{figure}

\end{block}
\end{column}

\end{columns}
\end{block}

\vspace{1em}
\begin{block}{Porovnání odhadů a měření synchronního trénování}

Úvahy ilustrované v předchozím bloku jsem použil k výpočtu odhadů zrychlení výpočtu na několika pracovních jednotkách.
To je zobrazené na následujících grafech.

\begin{columns}[t]

\begin{column}{0.2\linewidth}
\begin{block}{GoogLeNet}

\begin{figure}
\centering
\scalebox{2.0}{
\begin{tikzpicture}
\begin{axis}[domain=1:8,xtick={1,2,3,4,5,6,7,8},xlabel={Počet pracovních jednotek},ylabel={Zrychlení},label style={font=\tiny},tick label style={font=\tiny},ymin=0,ymax=8]
\addplot[color=black,mark=square*]coordinates{(1,0.6601502887)(2,1.3569755935)(3,1.6845214264)(4,1.7140744339)(5,1.744682906)(6,1.744682906)(7,1.744682906)(8,1.6845214264)};
\addplot[color=lightgray,mark=*]coordinates{(1,0.7027592832)(2,1.0797030714)(3,1.3314030678)(4,1.4649503391)(5,1.6090284145)(6,1.6759416177)(7,1.7347251018)(8,1.7528795351)};
\addplot[color=black,mark=square*,dashed]coordinates{(1,0.8074565515)(2,1.2855558254)(3,1.9157302497)(4,2.5051857111)(5,2.7914926495)(6,3.1516852495)(7,3.6186015827)(8,4.0709267806)};
\addplot[color=lightgray,mark=*,dashed]coordinates{(1,0.992248062)(2,1.569093763)(3,2.2307898125)(4,2.8329983603)(5,3.5570531207)(6,4.1814025589)(7,4.83492834)(8,5.4032627024)};
\end{axis}
\end{tikzpicture}
}
\end{figure}

\end{block}
\end{column}

\begin{column}{0.2\linewidth}
\begin{block}{SqueezeNet}

\begin{figure}
\centering
\scalebox{2.0}{
\begin{tikzpicture}
\begin{axis}[domain=1:8,xtick={1,2,3,4,5,6,7,8},xlabel={Počet pracovních jednotek},ylabel={Zrychlení},label style={font=\tiny},tick label style={font=\tiny},ymin=0,ymax=8]
\addplot[color=black,mark=square*]coordinates{(1,1.306122449)(2,2.4615384615)(3,3.4594594595)(4,4.1290322581)(5,4.9230769231)(6,5.5652173913)(7,5.8181818182)(8,6.4)};
\addplot[color=lightgray,mark=*]coordinates{(1,1.2291076898)(2,2.2981199353)(3,3.2838351138)(4,4.0569683372)(5,4.8952877237)(6,5.5134372866)(7,6.0974104024)(8,6.5219881601)};
\addplot[color=black,mark=square*,dashed]coordinates{(1,1.3473684211)(2,2.5098039216)(3,3.5555555556)(4,4)(5,4.7407407407)(6,5.5652173913)(7,5.8181818182)(8,6.4)};
\addplot[color=lightgray,mark=*,dashed]coordinates{(1,1.3095612077)(2,2.490303696)(3,3.7126663393)(4,4.8046936965)(5,6.0733903008)(6,7.1581776491)(7,8.272391696)(8,9.2322273034)};
\end{axis}
\end{tikzpicture}
}
\end{figure}

\end{block}
\end{column}

\begin{column}{0.45\linewidth}
\begin{block}{Resnet34}

\begin{figure}
\centering
\scalebox{2.0}{
\begin{tikzpicture}
\begin{axis}[domain=1:8,xtick={1,2,3,4,5,6,7,8},xlabel={Počet pracovních jednotek},ylabel={Zrychlení},legend pos=outer north east,label style={font=\tiny},tick label style={font=\tiny},legend style={font=\tiny},ymin=0,ymax=8]
\addplot[color=black,mark=square*]coordinates{(1,0.4508741315)(2,0.5332673737)(3,0.5558395377)(4,0.5332673737)(5,0.5558395377)(6,0.5415137764)(7,0.5252683631)(8,0.5252683631)};
\addplot[color=lightgray,mark=*]coordinates{(1,0.3997081134)(2,0.4966705391)(3,0.5464628319)(4,0.5597317406)(5,0.5852335211)(6,0.5880164777)(7,0.5921446289)(8,0.5855157104)};
\addplot[color=black,mark=square*,dashed]coordinates{(1,0.8337593066)(2,0.9550333875)(3,1.2359255603)(4,1.3643334108)(5,1.7508945438)(6,2.0202629352)(7,2.1010734526)(8,2.1439525026)};
\addplot[color=lightgray,mark=*,dashed]coordinates{(1,1)(2,1.0058906208)(3,1.3149228513)(4,1.6113233322)(5,1.9835271161)(6,2.3025694526)(7,2.6394072908)(8,2.9309359055)};
\legend{Centrální server - měření,Centrální server - výpočet,Distribuovaný server - měření,Distribuovaný server - výpočet}
\end{axis}
\end{tikzpicture}
}
\end{figure}

\end{block}
\end{column}

\end{columns}
\end{block}

\vspace{1em}
\begin{block}{Uložení vah}

Ve své práci jsem popsal dva způsoby, jak sdílet váhy neuronové sítě.
Váhy musí být sdíleny mezi všemi jednotkami, aby každá jednotka měla přístup k nejnovějším váhám, které použije pro výpočet gradientů.

\begin{columns}[t]

\begin{column}{0.51\linewidth}
\begin{block}{Centrální server}

Je mechanizmus uložení vah neuronové sítě na jedné dedikované jednotce.
Ostatní jednotky sdílejí váhy přes tento server.
Server provádí aktualizace vah sítě.
Problém je, že když více jednotek přenáší data ze serveru nebo na server, komunikace je zpomalená.

\begin{figure}
\centering
\scalebox{0.8}{
\begin{tikzpicture}[auto,x=1em,y=1em,z=1em,node distance=7em and 5em,>=triangle 45,align=left]
    \node [block_server](ps){Server s~parametry};
    \node [block_worker,below=of ps](w2){Pracovní\\ jednotka 2};
    \node [block_worker,left=of w2](w1){Pracovní\\ jednotka 1};
    \node [block_worker,right=of w2](w3){Pracovní\\ jednotka 3};
    \node [block_worker,right=of w3](w4){Pracovní\\ jednotka 4};
    
    \draw[->] ([xshift=10em]ps) -- node[]{$W$} (w1);
    \draw[->] ([xshift=5em]ps) -- node[]{$W$} (w2);
    \draw[->] ([xshift=7em]ps) -- node[]{$W$} (w3);
    \draw[->] ([xshift=10em]ps) -- node[]{$W$} (w4);
    
    \draw[->] ([xshift=-10em]w1) -- node[]{$\Delta W_1$} (ps);
    \draw[->] ([xshift=-5em]w2) -- node[]{$\Delta W_2$} (ps);
    \draw[->] ([xshift=-7em]w3) -- node[]{$\Delta W_3$} (ps);
    \draw[->] ([xshift=-10em]w4) -- node[]{$\Delta W_4$} (ps);
\end{tikzpicture}
}
\end{figure}

\end{block}
\end{column}

\begin{column}{0.45\linewidth}
\begin{block}{Distribuovaný server}

Cílem distribuovaného serveru je odstranit hlavní slabinu centrálního serveru.
Váhy sítě jsou rozděleny na $N$ částí, každá část je uložené na jiné jednotce.
Každá jednotka je tak server i výpočetní jednotka.

\vspace{1em}
\begin{figure}
\centering
\scalebox{0.8}{
\begin{tikzpicture}[auto,x=1em,y=1em,z=1em,node distance=7em and 7em,>=triangle 45,align=left]
    \node [block_worker](w1){Pracovní\\ jednotka 1};
    \node [block_worker,right=of w1](w2){Pracovní\\ jednotka 2};
    \node [block_worker,below=of w1](w3){Pracovní\\ jednotka 3};
    \node [block_worker,below=of w2](w4){Pracovní\\ jednotka 4};
    
    \draw[->,transform canvas={yshift=1em}] (w1) -- node[]{$W^1$} (w2);
    \draw[->,transform canvas={xshift=-1em}] (w1) -- node[]{$W^1$} (w3);
    \draw[->,transform canvas={xshift=1em}] (w1) -- node[xshift=-3em,yshift=2em]{$W^1$} (w4);
    
    \draw[->,transform canvas={yshift=0em}] (w2) -- node[]{$W^2$} (w1);
    \draw[->,transform canvas={xshift=1em}] (w2) -- node[xshift=3em,yshift=3em]{$W^2$} (w3);
    \draw[->,transform canvas={xshift=2em}] (w2) -- node[]{$W^2$} (w4);
    
    \draw[->,transform canvas={xshift=-2em}] (w3) -- node[]{$W^3$} (w1);
    \draw[->,transform canvas={xshift=-1em}] (w3) -- node[xshift=-2em,yshift=-2em]{$W^3$} (w2);
    \draw[->,transform canvas={yshift=0em}] (w3) -- node[]{$W^3$} (w4);
    
    \draw[->,transform canvas={xshift=-1em}] (w4) -- node[xshift=3em,yshift=-2em]{$W^4$} (w1);
    \draw[->,transform canvas={xshift=1em}] (w4) -- node[]{$W^4$} (w2);
    \draw[->,transform canvas={yshift=-1em}] (w4) -- node[]{$W^4$} (w3);
\end{tikzpicture}
}
\end{figure}

\end{block}
\end{column}

\end{columns}
\end{block}

\end{frame}
\end{document}

